
\subsection{Application}
The upward and downward formulation of the aforementioned equivalent densities and their translation are explained in detail in $\S3.2$ and shown in Figures 3, 4, 5, and 6.

In this section we use the Fourier representation of surface integral equations described in $\S2.2$ together with the 2D KIFMM to create a fast algorithm for densities distributed on axisymmetric surfaces of revolution. Notation in this section for the surfaces follows \cite{YBZ}.

Notice that $k_n$ does not depend only on the difference between each coordinate, e.g. $(r-r')$. This has implications for the translation operators we construct for the KIFMM which is discussed in $\S3.2.4$.

\subsection{Full algorithm}
Recall that in the 3D KIFMM, we needed to solve surface integral equations like:
\begin{align}
\mbox{S2M: }&\int_{\mathbf{y}^{B,u}}{K(\mathbf{x},\mathbf{y})}\phi^{B,u}{(\mathbf{y})}d\mathbf{y}=\sum\limits_{i\in I_s^B} K(\mathbf{x},\mathbf{y}_i)\phi_i\mbox{ for all }\mathbf{x}\in\mathbf{x}^{B,u}
\end{align}
We can write $(15)$ as a sequence of 2D equations, $(16)$, by using the Fourier representation explained in $\S2.2$. If this is unclear, equation $(15)$ is to $(16)$ as equation $(1)$ is to $(3)$. In calculations below, I will skip this derivation and just state the sequences of 2D equations.

Now that we have our kernel $k_n$, we proceed through the standard 2D KIFMM algorithm for the sources on the 2D generating curve $\gamma$ \textbf{for each} $n\in[-N,-N+1,\dots,N]$ where $N$ is the truncation parameter chosen earlier in $(10)$. The KIFMM algorithm is very similar to the FMM algorithm described in \cite{CGR}, apart from how the equivalent densities are represented, and how the translation operators are computed. As mentioned earlier, rather than by multipole expansions, the potential due to sources in a box is matched to an equivalent density at discretization points on a surface enclosing the box. In 2D, these surfaces are circles with radii prescribed in \cite{YBZ}. To compute these equivalent densities, we will need to discretize several integral operators on different surfaces, which is explained in $\S3.2.2$.

In the following equations, $\mathbf{x}=(r,z)$ and $\mathbf{y}=(r',z')$. Also, please note the seemingly out-of-place $r'$ term under each integral, and recall that this is actually part of the integral operator $K_n$ as in $(3)$.
\subsubsection{Equivalent densities}
After partitioning the hierarchical tree with no more than a prescribed number of sources in each box, compute the upward equivalent density for each leaf box. Similar to the \textbf{S2M} step in the FMM, solving the following equation for $\phi^{B,u}$ gives the upward equivalent density for a box $B$ in the KIFMM:
\begin{align}
\mbox{S2M: }&\int_{\mathbf{y}^{B,u}}{k_n(\mathbf{x},\mathbf{y})}\phi^{B,u}_n{(\mathbf{y})}r'd\mathbf{y}=\sum\limits_{i\in I_s^B} k_n(\mathbf{x},\mathbf{y}_i)\phi_ir'_i\mbox{ for all }\mathbf{x}\in\mathbf{x}^{B,u}\\
\mbox{Discretized S2M: }&M_n\phi^{B,u}_n=q_n^{B,u}
\end{align}
where in $(17)$, $M_n$ is the matrix of kernel evaluations of pairs of discretization points on the upward check surface and upward equivalent surface of a box $B$, $q_n^{B,u}$ is the upward check potential, and the densities at the actual source points in the box are $\phi_i$. This is similar to $(7)$, and this process is illustrated in Figure 3. Once the upward equivalent density is computed for each leaf box, we use translation operators to find the upward and downward equivalent densities for every other box.

\begin{figure}[!ht]
\begin{center}
%\includegraphics[scale=0.5]{ued-curve}
%\includegraphics[scale=0.5]{ded-curve}
\end{center}
\caption{Left: The upward check (blue) and equivalent (red) surfaces of a box used to compute the upward equivalent density due to source densities (black) in the box. Right: The downward check (blue) and equivalent (red) surfaces of a box used to compute the downward equivalent density due to sources densities (black) in the far field on the box. In the algorithm, we never actually directly compute a downward equivalent density.}
\end{figure}

\subsubsection{Translation Operators}

Translation operators limit the number of equivalent densities we need to compute directly by translating upward equivalent densities of the leaf boxes to upward and downward equivalent densities of all other boxes. In the KIFMM in particular, the translation operators in their discretized form are matrices that translate an equivalent density from one box to that of another. They are a pre-computation, as they are constant regardless of the source distribution.

In the previous step, we computed the upward equivalent density for each leaf box. The \textbf{M2M} operator translates the upward equivalent density from a leaf box $A$ to the upward equivalent density of its parent box $B$. Solving the following equation for $\phi_n^{B,u}$ gives the M2M operator.
\begin{align}
\mbox{M2M: }&\int_{\mathbf{y}^{B,u}}{k_n(\mathbf{x},\mathbf{y})}\phi^{B,u}_n{(\mathbf{y})}r'd\mathbf{y} = \int_{\mathbf{y}^{A,u}}{k_n(\mathbf{x},\mathbf{y})}\phi^{A,u}_n{(\mathbf{y})}r'd\mathbf{y}\mbox{ for all }\mathbf{x}\in\mathbf{x}^{B,u}\\
\mbox{Discretized M2M: }&M_n^B\phi^{B,u}_n=M_n^A\phi^{A,u}_n
\end{align}
where $M_n^A$ is the matrix of kernel evaluations of pairs of discretization points on the upward check surface of box $B$ and the upward equivalent surface of box $A$, and $M_n^B$ is the matrix of kernel evaluations of pairs of discretization points on the upward check surface of box $B$ and the upward equivalent surface of box $B$. Figure 4 gives a graphical representation of this step.

So the M2M translation operator is:

\begin{align}
T^{M2M} &= (M_n^B)^{-1}M_n^A\\
&= \begin{pmatrix}
  K(\mathbf{x_1},\mathbf{y_1}) & \cdots & K(\mathbf{x_1},\mathbf{y_m})  \\
  \vdots  & \ddots & \vdots  \\
  K(\mathbf{x_m},\mathbf{y_1}) & \cdots & K(\mathbf{x_m},\mathbf{y_m}) 
 \end{pmatrix}^{-1}\begin{pmatrix}
  K(\mathbf{x_1},\mathbf{y_1}) & \cdots & K(\mathbf{x_1},\mathbf{y_m})  \\
  \vdots  & \ddots & \vdots  \\
  K(\mathbf{x_m},\mathbf{y_1}) & \cdots & K(\mathbf{x_m},\mathbf{y_m}) 
 \end{pmatrix}
\end{align}


\begin{figure}[!ht]
\begin{center}
%\includegraphics[scale=0.5]{M2M-curve}
\end{center}
\caption{The upward equivalent density due to the source densities (black) in the child box computed from the upward equivalent (red) and check (blue) surfaces of the child box is translated to the upward equivalent density of the parent box, checking against its upward check (blue) and equivalent (red) surfaces.}
\end{figure}

After repeating that step, every box now has an upward equivalent density. The \textbf{M2L} operator translates the upward equivalent density from a non-leaf box $A$ to the downward equivalent density of a box $B$ on the same level.
\begin{align}
\mbox{M2L: }&\int_{\mathbf{y}^{B,d}}{k_n(\mathbf{x},\mathbf{y})}\phi^{B,d}_n{(\mathbf{y})}r'd\mathbf{y} = \int_{\mathbf{y}^{A,u}}{k_n(\mathbf{x},\mathbf{y})}\phi^{A,u}_n{(\mathbf{y})}r'd\mathbf{y}\mbox{ for all }\mathbf{x}\in\mathbf{x}^{B,d}\\
\mbox{Discretized M2L: }&M_n^B\phi^{B,d}_n=M_n^A\phi^{A,u}_n
\end{align}
where $M_n^A$ is the matrix of kernel evaluations of pairs of discretization points on the downward check surface of box $B$ and the upward equivalent surface of box $A$, and $M_n^B$ is the matrix of kernel evaluations of pairs of discretization points on the downward check surface of box $B$ and the downward equivalent surface of box $B$.

\begin{figure}[!ht]
\begin{center}
%\includegraphics[scale=0.5]{M2L-curve}
\end{center}
\caption{The upward equivalent density due to the source densities (black) in a box computed from the upward equivalent (red) and check (blue) surfaces of the child box is translated to the downward equivalent density of another box on the same level, checking against its upward check (blue) and equivalent (red) surfaces.}
\end{figure}

After repeating that step, every non-leaf box has a downward equivalent density. The \textbf{L2L} operator translates the downward equivalent density of a non-leaf box $A$ to the downward equivalent density of a child box $B$.
\begin{align}
\mbox{L2L: }&\int_{\mathbf{y}^{B,d}}{k_n(\mathbf{x},\mathbf{y})}\phi^{B,d}_n{(\mathbf{y})}r'd\mathbf{y}=\int_{\mathbf{y}^{A,d}}{k_n(\mathbf{x},\mathbf{y})}\phi^{A,d}_n{(\mathbf{y})}r'd\mathbf{y}\mbox{ for all }\mathbf{x}\in\mathbf{x}^{B,d}\\
\mbox{Discretized L2L: }&M_n^B\phi^{B,d}_n=M_n^A\phi^{A,d}_n
\end{align}
where $M_n^A$ is the matrix of kernel evaluations of pairs of discretization points on the downward check surface of box $B$ and the downward equivalent surface of box $A$, and $M_n^B$ is the matrix of kernel evaluations of pairs of discretization points on the downward check surface of box $B$ and the downward equivalent surface of box $B$. After repeating this step, every box now has an upward and downward equivalent density.

\begin{figure}[!ht]
\begin{center}
%\includegraphics[scale=0.5]{L2L-curve}
\end{center}
\caption{The downward equivalent density due to the source densities (black) in a parent box computed from the upward equivalent (red) and check (blue) surfaces of the parent box is translated to the downward equivalent density of a child box, checking against its upward check (blue) and equivalent (red) surfaces.}
\end{figure}

Once we have these upward and downward equivalent densities for every $n\in[-N,\dots,N]$, then we can reconstruct the 3D equivalent densities $\phi_{approx}$ by summing as in $(10)$. This summing can be accelerated via the FFT. Having these equivalent densities in 3D, we can complete the last step of the KIFMM algorithm by evaluating the far- and then near-field interactions.

Note that all requirements for smoothness and uniqueness of the solution to the integral equations in the KIFMM listed in $\S3.1.5$ of \cite{YBZ} (mostly pertaining to the sizes and positions of the surfaces) are satisfied here.

\subsubsection{Discretization details}
The equations in $\S3.2.1$ are discretized using $p$ points on the equivalent and check surfaces. $p$ is constant for all boxes, and the choice of $p$ determines the error level of the computations.This discretization requires two steps. First, the right hand side is the evaluation of a check potential. This step checks that the potential represented by the equivalent density and the actual source densities are the same to all boxes in the far field. Then on the left hand side, we invert a Dirichlet-type boundary integral equation to obtain the equivalent density. Essentially, each translation is simply applying a series of matrices.

For example, the M2L operator is the matrix $T^{M2L}_n$ for a mode $n$ is obtained by solving $(20)$:
\begin{align}
\phi^{B,d}_n&=\bigg[\big[\alpha I +(M_n^B)^TM_n^B\big](M_n^B)^TM_n^A\bigg]\phi^{A,u}_n\\
\implies T^{M2L}_n &= \bigg[\big[\alpha I +(M_n^B)^TM_n^B\big](M_n^B)^TM_n^A\bigg]
\end{align}

