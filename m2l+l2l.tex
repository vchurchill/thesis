
$\mathbf{2D-M2L+L2L.py}$

This file compares the actual local expansion of a box with our estimate of its local expansion. In this example we consider the computational domain to be only these boxes, with empty space (no sources) outside it.

The computation of the actual local expansion is straightforward. This is the potential at a target point inside the target box due to all sources in the far field $\mathbf{F}$. For our example, this means a target point in the box $T$ due to sources in $P_1,\dots,P_7$ and $C_1,\dots,C_{27}$. For the target point $\mathbf{x_0}$ and source points $\mathbf{x_j}$ with weight $\sigma_j$, the calculation is:

\begin{align}
f(\mathbf{x_0}) = \sum_{\mathbf{x_j}\in \mathbf{F}} K(\mathbf{x_0},\mathbf{x_j})\sigma_j
\end{align}

Now we compute the estimated local expansion. In general in an FMM, the local expansion for a box is the sum of the multipole-to-local operations and the local-to-local operation for this box.

We start by placing sources points in each box $P_1,\dots,P_7$ and $C_1,\dots,C_{27}$, and assigning each source a charge of $+1$ or $-1$. The $C$ boxes are in the interaction list of the target box. The $P$ boxes are in the interaction list of the parent box of the target box. We need to develop an equivalent density for each box $P_1,\dots,P_7$ and $C_1,\dots,C_{27}$, which is a set of $n^2$ weights at the Chebyshev nodes in each box. For each box, take $P1$ for example, this equivalent density $W$ at a node $\overline{y}_{m_1,m_2}$ is the sum over $\mathbf{y}_j$ in $P_1$ of the $R$ function at each node and each source multiplied by the charge at the source:
\begin{align}
W(m_1,m_2) = \sum_{y_j\in P_1} R_n(\mathbf{\overline{y}}_{m_1,m_2},\mathbf{y}_j)\sigma_j
\end{align}
Once these weights have been computed for every box in the far field, we compute the multipole-to-local operations for each of the boxes in the interaction list of the parent box. The sum of these over each interaction list box will be the entire local expansion for the parent, since the contribution of the local-to-local operator for this parent is $0$ as the grandparent in our case has no interaction list in the computational domain. The $M2L$ operation at each node in the parent box is such for parent box $P$ and interaction boxes $P_1,\dots,P_7$:
\begin{align}
g^P_{l_1,l_2} = \sum_{i=1}^7\sum_{m_1=1}^n\sum_{m_2=1}^n W_{m_1,m_2}^{P_i}K(\mathbf{\overline{x}}^P_{l_1,l_2},\mathbf{\overline{y}}^{P_i}_{m_1,m_2})
\end{align}

Now we compute the same $M2L$ operations at each node in the target box $T$:
\begin{align}
g^T_{l_1,l_2} = \sum_{i=1}^{27}\sum_{m_1=1}^n\sum_{m_2=1}^n W_{m_1,m_2}^{C_i}K(\mathbf{\overline{x}}^C_{l_1,l_2},\mathbf{\overline{y}}^{C_i}_{m_1,m_2})
\end{align}

Then we compute the estimated local expansion $f$ at each node in the box $T$:
\begin{align}
f^T_{l_1,l_2} = g^T_{l_1,l_2} + \sum_{l'_1=1}^n\sum_{l'_2=1}^n g^P_{l'_1,l'_2}R_n(\mathbf{\overline{x}}^T_{l_1,l_2},\mathbf{\overline{x}}^P_{l'_1,l'_2})
\end{align}

Finally we compute the estimated potential at the target point due to sources in the far field $\mathbf{F}$:
\begin{align}
f(\mathbf{x}_0) = \sum_{l_1=1}^n\sum_{l_2=1}^n f^T_{l_1,l_2} R_n(\mathbf{\overline{x}}^T_{l_1,l_2},\mathbf{x}_0)
\end{align}

Compare this with the naive computation of the potential.

