%% NYU PhD thesis format. Created by Jos� Koiller 2007--2008.

%% Use the first of the following lines during production to
%% easily spot "overfull boxes" in the output. Use the second
%% line for the final version.
%\documentclass[12pt,draft,letterpaper]{report}
\documentclass[12pt,letterpaper]{report}

%% Replace the title, name, advisor name, graduation date and dedication below with
%% your own. Graduation months must be January, May or September.
\newcommand{\thesistitle}{Translation Operators for Modal Green's Functions}
\newcommand{\thesisauthor}{Victor Churchill}
\newcommand{\thesisadvisor}{Professor Michael O'Neil}
\newcommand{\graddate}{May 2016}
%% If you do not want a dedication, scroll down and comment out
%% the appropriate lines in this file.
%\newcommand{\thesisdedication}{To me.}

%% The following makes chapters and sections, but not subsections,
%% appear in the TOC (table of contents). Increase to 2 or 3 to
%% make subsections or subsubsections appear, respectively. It seems
%% to be usual to use the "1" setting, however.
\setcounter{tocdepth}{1}

%% Sectional units up to subsubsections are numbered. To number
%% subsections, but not subsubsections, decrease this counter to 2.
\setcounter{secnumdepth}{3}

%% Page layout (customized to letter paper and NYU requirements):
\setlength{\oddsidemargin}{.6in}
\setlength{\textwidth}{5.8in}
\setlength{\topmargin}{.1in}
\setlength{\headheight}{0in}
\setlength{\headsep}{0in}
\setlength{\textheight}{8.3in}
\setlength{\footskip}{.5in}

%% Use the following commands, if desired, during production.
%% Comment them out for final version.
%\usepackage{layout} % defines the \layout command, see below
%\setlength{\hoffset}{-.75in} % creates a large right margin for notes and \showlabels

%% Controls spacing between lines (\doublespacing, \onehalfspacing, etc.):
\usepackage{setspace}

%% Use the line below for official NYU version, which requires
%% double line spacing. For all other uses, this is unnecessary,
%% so the line can be commented out.
\doublespacing % requires package setspace, invoked above

%% Each of the following lines defines the \com command, which produces
%% a comment (notes for yourself, for instance) in the output file.
%% Example:    \com{this will appear as a comment in the output}
%% Choose (uncomment) only one of the three forms:
%\newcommand{\com}[1]{[/// {#1} ///]}       % between [/// and ///].
\newcommand{\com}[1]{\marginpar{\tiny #1}} % as (tiny) margin notes
%\newcommand{\com}[1]{}                     % suppress all comments.

%% This inputs your auxiliary file with \usepackage's and \newcommand's:
%% It is assumed that that file is called "definitions.tex".
%%%
%% Place here your \usepackage's. Some recommended packages are already included.
%%

% Graphics:
\usepackage[final]{graphicx}
%\usepackage{graphicx} % use this line instead of the above to suppress graphics in draft copies
%\usepackage{graphpap} % \defines the \graphpaper command

% Indent first line of each section:
\usepackage{indentfirst}

% Good AMS stuff:
\usepackage{amsthm} % facilities for theorem-like environments
\usepackage[tbtags]{amsmath} % a lot of good stuff!

% Fonts and symbols:
\usepackage{amsfonts}
\usepackage{amssymb}

% Formatting tools:
%\usepackage{relsize} % relative font size selection, provides commands \textsmalle, \textlarger
%\usepackage{xspace} % gentle spacing in macros, such as \newcommand{\acims}{\textsc{acim}s\xspace}

% Page formatting utility:
%\usepackage{geometry}

%%
%% Place here your \newcommand's and \renewcommand's. Some examples already included.
%%
\renewcommand{\le}{\leqslant}
\renewcommand{\ge}{\geqslant}
\renewcommand{\emptyset}{\ensuremath{\varnothing}}
\newcommand{\ds}{\displaystyle}
\newcommand{\R}{\ensuremath{\mathbb{R}}}
\newcommand{\Q}{\ensuremath{\mathbb{Q}}}
\newcommand{\Z}{\ensuremath{\mathbb{Z}}}
\newcommand{\N}{\ensuremath{\mathbb{N}}}
\newcommand{\T}{\ensuremath{\mathbb{T}}}
\newcommand{\eps}{\varepsilon}
\newcommand{\closure}[1]{\ensuremath{\overline{#1}}}
%\newcommand{\acim}{\textsc{acim}\xspace}
%\newcommand{\acims}{\textsc{acim}s\xspace}

%%
%% Place here your \newtheorem's:
%%

%% Some examples commented out below. Create your own or use these...
%%%%%%%%%\swapnumbers % this makes the numbers appear before the statement name.
%\theoremstyle{plain}
%\newtheorem{thm}{Theorem}[chapter]
%\newtheorem{prop}[thm]{Proposition}
%\newtheorem{lemma}[thm]{Lemma}
%\newtheorem{cor}[thm]{Corollary}

%\theoremstyle{definition}
%\newtheorem{define}{Definition}[chapter]

%\theoremstyle{remark}
%\newtheorem*{rmk*}{Remark}
%\newtheorem*{rmks*}{Remarks}

%% This defines the "proo" environment, which is the same as proof, but
%% with "Proof:" instead of "Proof.". I prefer the former.
%\newenvironment{proo}{\begin{proof}[Proof:]}{\end{proof}}

%%
%% Place here your \usepackage's. Some recommended packages are already included.
%%

% Graphics:
\usepackage[final]{graphicx}
%\usepackage{graphicx} % use this line instead of the above to suppress graphics in draft copies
%\usepackage{graphpap} % \defines the \graphpaper command

% Indent first line of each section:
\usepackage{indentfirst}

% Good AMS stuff:
\usepackage{amsthm} % facilities for theorem-like environments
\usepackage[tbtags]{amsmath} % a lot of good stuff!

% Fonts and symbols:
\usepackage{amsfonts}
\usepackage{amssymb}

% Formatting tools:
%\usepackage{relsize} % relative font size selection, provides commands \textsmalle, \textlarger
%\usepackage{xspace} % gentle spacing in macros, such as \newcommand{\acims}{\textsc{acim}s\xspace}

% Page formatting utility:
%\usepackage{geometry}

%%
%% Place here your \newcommand's and \renewcommand's. Some examples already included.
%%
\renewcommand{\le}{\leqslant}
\renewcommand{\ge}{\geqslant}
\renewcommand{\emptyset}{\ensuremath{\varnothing}}
\newcommand{\ds}{\displaystyle}
\newcommand{\R}{\ensuremath{\mathbb{R}}}
\newcommand{\Q}{\ensuremath{\mathbb{Q}}}
\newcommand{\Z}{\ensuremath{\mathbb{Z}}}
\newcommand{\N}{\ensuremath{\mathbb{N}}}
\newcommand{\T}{\ensuremath{\mathbb{T}}}
\newcommand{\eps}{\varepsilon}
\newcommand{\closure}[1]{\ensuremath{\overline{#1}}}
%\newcommand{\acim}{\textsc{acim}\xspace}
%\newcommand{\acims}{\textsc{acim}s\xspace}

%%
%% Place here your \newtheorem's:
%%

%% Some examples commented out below. Create your own or use these...
%%%%%%%%%\swapnumbers % this makes the numbers appear before the statement name.
%\theoremstyle{plain}
%\newtheorem{thm}{Theorem}[chapter]
%\newtheorem{prop}[thm]{Proposition}
%\newtheorem{lemma}[thm]{Lemma}
%\newtheorem{cor}[thm]{Corollary}

%\theoremstyle{definition}
%\newtheorem{define}{Definition}[chapter]

%\theoremstyle{remark}
%\newtheorem*{rmk*}{Remark}
%\newtheorem*{rmks*}{Remarks}

%% This defines the "proo" environment, which is the same as proof, but
%% with "Proof:" instead of "Proof.". I prefer the former.
%\newenvironment{proo}{\begin{proof}[Proof:]}{\end{proof}}

%% Cross-referencing utilities. Use one or the other--whichever you prefer--
%% but comment out both lines for final version.
%\usepackage{showlabels}
%\usepackage{showkeys}


\begin{document}
%% Produces a test "layout" page, for "debugging" purposes only.
%% Comment out for final version.
%\layout % requires package layout (see above, on this same file)

%%%%%% Title page %%%%%%%%%%%
%% Sets page numbering to "roman style" i, ii, iii, iv, etc:
\pagenumbering{roman}
%
%% No numbering in the title page:
\thispagestyle{empty}
%
\begin{center}
  {\large\textbf{\thesistitle}}
  \vspace{.7in}

  by
  \vspace{.7in}

  \thesisauthor
  \vfill

\begin{doublespace}
  A thesis submitted in partial fulfillment\\
  of the requirements for the degree of\\
  Master of Science\\
  Department of Mathematics\\
  New York University\\
  \graddate
\end{doublespace}
\end{center}
\vfill

\noindent\makebox[\textwidth]{\hfill\makebox[2.5in]{\hrulefill}}\\
\makebox[\textwidth]{\hfill\makebox[2.5in]{\hfill\thesisadvisor\hfill}}
%\newpage
%%%%%%%%%%%%% Blank page %%%%%%%%%%%%%%%%%%
\thispagestyle{empty}
\vspace*{0in}
%\newpage

%%%%%%%%%%%%%% Dedication %%%%%%%%%%%%%%%%%
%% Comment out the following lines if you do not want to dedicate
%% this to anyone...
%\vspace*{\fill}
%\begin{center}
%  \thesisdedication\addcontentsline{toc}{section}{Dedication}
%\end{center}
%\vfill
%\newpage
%%%%%%%%%%%%%% Acknowledgements %%%%%%%%%%%%
%% Comment out the following lines if you do not want to acknowledge
%% anyone's help...
%\section*{Acknowledgements}\addcontentsline{toc}{section}{Acknowledgements}
%%% Write your acknowledgements in this file. If you do not want to acknowledge anyone,
%% you can delete this file and comment out the corresponding part in the "thesis.tex"
%% file.
%
I would like to acknowledge the effort put into this thesis by
myself. This work could not have been done without me.

%I would like to acknowledge the effort put into this thesis by myself. This work could not have been done without me.
%\newpage
%%%% Abstract %%%%%%%%%%%%%%%%%%
%\section*{Abstract}\addcontentsline{toc}{section}{Abstract}
%%% Write your abstract here.
%
Translation operators for various flavors of the Fast Multipole Method (FMM) are a minimal pre-computation for translation-invariant Green's functions and modal Green's functions. This is due to the fact that for these functions many translation operators on various levels the computational domain are the same, and so need not be repeatedly computed. However, for a translation-variant Green's function or modal Green's function this is not immediately clear. Until now one may have needed to compute all possible translation operators between all boxes in the computational domain. This thesis shows that lightening the pre-computational load of the translation operators for the FMM is possible for translation-variant modal Green's functions. Specifically we examine the application to the translation operators of the three-dimensional kernel-independent FMM for Laplace's equation.

%Translation operators for various flavors of the Fast Multipole Method (FMM) are a minimal pre-computation for translation-invariant Green's functions and modal Green's functions. This is due to the fact that for these functions many translation operators on various levels the computational domain are the same, and so need not be repeatedly computed. However, for a translation-variant Green's function or modal Green's function this is not immediately clear. Until now one may have needed to compute all possible translation operators between all boxes in the computational domain. This thesis shows that lightening the pre-computational load of the translation operators for the FMM is possible for translation-variant modal Green's functions. Specifically we examine the application to the translation operators of the three-dimensional kernel-independent FMM for Laplace's equation.
%\newpage
%%%% Table of Contents %%%%%%%%%%%%
\tableofcontents

%%%%% List of Figures %%%%%%%%%%%%%
%% Comment out the following two lines if your thesis does not
%% contain any figures. The list of figures contains only
%% those figures included withing the "figure" environment.
\listoffigures\addcontentsline{toc}{section}{List of Figures}
\newpage

%%%%% List of Tables %%%%%%%%%%%%%
%% Comment out the following two lines if your thesis does not
%% contain any tables. The list of tables contains only
%% those tables included withing the "table" environment.
\listoftables\addcontentsline{toc}{section}{List of Tables}
\newpage

%%%%% Body of thesis starts %%%%%%%%%%%%
\pagenumbering{arabic} % switches page numbering to arabic: 1, 2, 3, etc.
%% Introduction. If your thesis has no introduction, or chapter 1 is
%% meant to be the introduction, then comment out the lines below.
%\section*{Introduction}\addcontentsline{toc}{section}{Introduction}
%%% Write your introduction here.
%
This thesis is about the Riemann Hypothesis. We provide an
affirmative answer to the question ``If $z$ is a zero of the
zeta function and $0\le \Re(z)\le1$, then is $\Re(z)$
necessarily equal to $1/2$?''

In chapter~\ref{chap:one} we do this and that.

In the latter chapters we\ldots

%This thesis is about translation operators.

%In chapter~\ref{chap:one}
%% If your thesis has different "Parts", use commands such as the following:
%\part{First Part\label{part:one}}%

I should do multiple computational domains with multiple numbers of levels and time them for comparison.

Section 1 is devoted to the FMM and FMM-accelerated direct fast solvers

Section 2 is devoted to the KIFMM

Section 3 KIFMM with modal green's functions (Martinsson, Young, Hao)
Its application to our problem is difficult because the pre-computation will be extremely expensive. There isn't any constant proportionality between the operators and so each operator will have to be computed individually which would be O().

Section 4 bbFMM review

Section 5 bbFMM applied to 


% CHAPTER 1!
\chapter{Statement of problem\label{chap:one}}

In this chapter, \ldots

\section{Background\label{sec:hypothesis}}

Electromagnetic scattering by a body of revolution is a significant problem with radar, geophysical exploration, and acoustics applications.

reduce the problem from three dimensions to a series of problems in two dimensions.

efficient evaluation of modal green's function was explored by Abdelmageed (see equation $39$), and others. These could be implemented to actually compute the kernel at different Chebyshev nodes.

We are, however, concerned not only with the efficient evaluation of these analytic expansions, but also the fast evaluation of their sums.

The goal of this project is to formulate translation operators for a fast multipole method for use with modal kernels.

In this project, we work in three-cylindrical coordinates $(r,\theta,z)$ such that a point in Cartesian coordinates $(x,y,z)$ is represented by:

\begin{align}
r &= \sqrt{x^2+y^2}\\
\theta &=   \begin{cases}
    0 & \mbox{if } x = 0 \mbox{ and } y = 0\\
    \arcsin(\frac{y}{r}) & \mbox{if } x \geq 0 \\	
    \arctan(\frac{y}{x}) & \mbox{if } x > 0 \\	
    -\arcsin(\frac{y}{r}) + \pi & \mbox{if } x < 0
  \end{cases}\\
z &= z
\end{align}

A kernel is called translation invariant if for two points $(r,\theta,z)$ and $(r',\theta',z')$:
\begin{align}
K(r,\theta,z,r',\theta',z')=K(r-r',\theta-\theta',z-z')
\end{align}

\begin{align}
K(\mathbf{x},\mathbf{x'})=\sum_{n\in\mathbb{Z}} \frac{e^{in(\theta-\theta')}}{\sqrt{2\pi}}k_n(r,z,r',z')
\end{align}

for $\mathbf{x}=(r,z,\theta)$ and $\mathbf{x'}=(r',z',\theta')$, where the $k_n$ functions are called Fourier modes. When a kernel is as above, a function of $(\theta-\theta',r,z,r',z')$, it is called rotationally symmetric, or axisymmetric. In particular, we are looking at the Fourier modes of the single-layer three-dimensional Laplace kernel:

\begin{align}
\frac{1}{4\pi |\mathbf{x}-\mathbf{x'}|}=\sum_{n\in\mathbb{Z}} \frac{e^{in(\theta-\theta')}}{\sqrt{2\pi}} s_n(r,z,r',z')
\end{align}
with: 
\begin{align}
s_n(r,z,r',z')=\frac{1}{\sqrt{8\pi^3 rr'}}\mathbf{Q}_{n-\frac{1}{2}}\bigg(\frac{r^2+(r')^2+(z-z')^2}{2rr'}\bigg)
\end{align}

This is specifically for particles or charges governed by an interior or exterior Dirichlet problem with Laplace's equation. On a three-dimensional surface $\Gamma$,

\begin{align}
\Delta u &= 0 \mbox{ in }\Omega\\
u &= f \mbox{ on }\Gamma
\end{align}

We're interested in a more general class of density distributions.

This tactic of reducing a three-dimensional algorithm to a series of two dimensional algorithms is explained at length by Martinsson, Young, and Hao.

Translation operators translate weights or densities in one box to another box in the computational domain. Computing every translation operator in the entire computational domain would be extremely expensive. Consider one of the original uses of the FMM, computing pairwise forces between stars and planets in space, and this is clear.

If a kernel is translation invariant, then translation operators will only differ based on relative position and level in the hierarchical tree. This is because the kernels used there only depended the relative difference between coordinate values, e.g. $(r-r')$.

Translation operators can be computed without any knowledge of the sources or targets, so they are a true pre-computation. The only thing we'll need is a computational domain. We choose arbitrarily the box $[-1,1]\times[-1,1]$, although operators can be calculated for any rectangle.

There are three types of translation operators. The multipole-to-multipole (M2M) operator translates densities in a box to densities in the parent box. The multipole-to-local (M2L) operator translates densities in a box to another box on the same level in the hierarchical tree that divides the computational domain. The local-to-local (L2L) operator translates densities in a box to densities in a child box.

Translation operators are the backbone of a fast multipole method.

These translation operators depend on the kernel of the partial differential equation that governs the relationship between source and target particles.

The fast multipole method (FMM) has been applied to many different kernels. It has also been applied to many different density distributions. There are different flavors of the FMM to deal with difficult kernels for which there aren't workable analytic expansions. These so-called kernel-independent FMMs get around this issue. The original KIFMM uses a continuous distribution of an equivalent density on a surface enclosing a box in the hierarchical tree to represent the potential generated by sources in that box, rather than using analytic multipole expansions as in the original FMM. This allows us to construct an efficient FMM that only requires kernel evaluations. The KIFMM is also relatively easy to implement, since in general it applies to an arbitrary kernel that is the fundamental solution of some elliptic PDE. To change the kernel in the original FMM, one would need to develop analytic multipole expansions for that kernel that may be difficult to produce.

The upward and downward formulation of the aforementioned equivalent densities and their translation are explained in detail in $\S3.2$ and shown in Figures 3, 4, 5, and 6.

\begin{figure}[!ht]
\begin{center}
%\includegraphics[scale=0.9]{3dsurface}
\end{center}
\caption{An axisymmetric surface $\Gamma$ in 3D.}
\end{figure}

Later, Fong and Darve proposed another kernel-independent FMM, the black-box FMM (bbFMM). This approach uses 

\newpage

\section{Contribution of this project}

The original KIFMM paper by Ying, Biros, and Zorin, explored three data sets for the 3D case: densities distributed on the unit sphere, densities distributed uniformly on the unit cube, and densities distributed at the eight corners of the unit cube.

This project examines the case of a broader category of non-uniform distributions, surfaces of revolution that are rotationally symmetric with respect to the azimuthal angle, or axisymmetric, as in Figure 1. While the original 3D KIFMM may be able to handle densities distributed on surfaces of revolution, \cite{YBZ} explains that the implementation is difficult. This project proposes to work around this issue by employing the technique described in \cite{YYM} to replace the 3D integral equations required by the 3D KIFMM with their Fourier representations, sequences of 2D integral equations. In this way, we can avoid the 3D KIFMM in favor of repeatedly applying the 2D KIFMM to accelerate pairwise computations of densities distributed on an axisymmetric surfaces of revolution.

\section{Background}

\subsection{Fourier representation of 3D integral equations}
Our strategy to apply the KIFMM to densities distributed on axisymmetric surfaces of revolution is grounded in the fact that it is easier to solve boundary integral equations defined on curves in $\mathbb{R}^2$ than those defined on surfaces in $\mathbb{R}^3$. This section is based on results in \cite{YYM}.

Consider the Fredholm integral equation of the first kind defined on the axisymmetric surface $\Gamma$ in 3D:
\begin{align}
\int_\Gamma k(\mathbf{x},\mathbf{y})\phi(\mathbf{y})d\mathbf{y} &= q(\mathbf{x}) &\mathbf{x}\in\Gamma
\end{align}
where $k$ is a kernel function, $\phi$ is an unknown density, and $q$ is a potential. This is exactly the same type of integral equation we see in the 3D KIFMM.

The surface $\Gamma$ is obtained by rotating a curve $\gamma$ about the $z$ axis. $\gamma$ is called the generating curve, shown in Figure 2. In particular, $\Gamma=\gamma\times\mathbb{T}$ where $\mathbb{T}$ is the one-dimensional torus (circle) parametrized by $\theta\in(-\pi,\pi]$. Since $k(\mathbf{x},\mathbf{y})$ is axisymmetric, we have that $k$ is a function only of the difference between $\theta$ and $\theta'$:
\begin{align}
k(\mathbf{x},\mathbf{y})=k(\theta-\theta',r,z,r',z')
\end{align}
where $\mathbf{x}=(r,z,\theta)$ and $\mathbf{y}=(r',z',\theta')$  in 3D cylindrical coordinates.

\begin{figure}[!ht]
\begin{center}
%\includegraphics[scale=0.4]{2dcurve}
\end{center}
\caption{The generating curve $\gamma$ in 2D, of an axisymmetric surface $\Gamma$ in 3D.}
\end{figure}

Due to these convenient circumstances, we can restate $(1)$ as a sequence of integral equations defined on the generating curve by performing a Fourier transform. If $\phi_n$, $q_n$, and $k_n$ are the Fourier modes of $k$, $\phi$, and $q$, then $(1)$ becomes:
\begin{align}
\sqrt{2\pi}\int_\gamma k_n(r,z,r',z')\phi_n(r',z')r'dl(r',z')&=q_n(r,z) &(r,z)\in\gamma, n\in\mathbb{Z}
\end{align}
where we have:
\begin{align}
q_n(r,z)&=\int_\mathbb{T}\frac{e^{-in\theta}}{\sqrt{2\pi}}q(r,z,\theta)d\theta&q(\mathbf{x})&=\sum_{n\in\mathbb{Z}}\frac{e^{in\theta}}{\sqrt{2\pi}}q_n(r,z)\\
\phi_n(r,z)&=\int_\mathbb{T}\frac{e^{-in\theta}}{\sqrt{2\pi}}\phi(r,z,\theta)d\theta&\phi(\mathbf{x})&=\sum_{n\in\mathbb{Z}}\frac{e^{in\theta}}{\sqrt{2\pi}}\phi_n(r,z)\\
k_n(r,z,r',z')&=\int_\mathbb{T}\frac{e^{-in\theta}}{\sqrt{2\pi}}k(r,z,r',z',\theta)d\theta&k(\mathbf{x},\mathbf{y})&=\sum_{n\in\mathbb{Z}}\frac{e^{in\theta}}{\sqrt{2\pi}}k_n(r,z,r',z')
\end{align}
In $\S3.2$, we'll apply this same transformation to the 3D integral equations prescribed in the 3D KIFMM, approximating with sequences of 2D integral equations that are easier to solve. For convenience, we can rewrite $(3)$ as:
\begin{align}
K_n\phi_n=q_n
\end{align}
If $K_n$ is a continuously invertible operator, we have:
%As in [2], we use Tikhonov regularization for a stable solve.
\begin{align}
\phi_n=K_n^{-1}q_n
%\phi_n=[\alpha I+K_n^TK_n]K_n^Tq_n
\end{align}
When we implement this strategy in combination with the KIFMM, we'll use Tikhonov regularization to stably solve $(8)$. Plugging in to $(6)$ gives the solution of the original 3D surface integral equation $(1)$:
\begin{align}
\phi(r,z,\theta)=\sum_{n\in\mathbb{Z}}\frac{e^{in\theta}}{\sqrt{2\pi}}[K_n^{-1}q_n](r,z)
%\phi(r,z,\theta)=\sum_{n\in\mathbb{Z}}\frac{e^{in\theta}}{\sqrt{2\pi}}\big[(\alpha I + K_n^TK_n)^{-1}K_n^Tq_n\big](r,z)
\end{align}
In practice, we choose a truncation parameter, $N$, such that $||q-\sum\limits_{n=-N}^{N}\frac{e^{in\theta}}{\sqrt{2\pi}}q_n||\le\epsilon$:
\begin{align}
\phi_{approx}=\sum_{n=-N}^N\frac{e^{in\theta}}{\sqrt{2\pi}}K_n^{-1}q_n
\end{align}

\section{Application of the KIFMM}
In this section we use the Fourier representation of surface integral equations described in $\S2.2$ together with the 2D KIFMM to create a fast algorithm for densities distributed on axisymmetric surfaces of revolution. Notation in this section for the surfaces follows \cite{YBZ}.

\subsection{Single-layer 3D Laplace kernel}
For now, we apply the algorithm considering only the single-layer 3D Laplace kernel. This choice has been made because its Fourier modes can be solved for analytically. For many other kernels, that is not that case, and we would need to approximate the modes using a discretization. This is left for future work.

We find the Fourier modes for this kernel below as in \cite{YYM}. For $\mathbf{x}=(r,z,\theta)$ and $\mathbf{y}=(r',z',\theta')$:
\begin{align}
k(\mathbf{x},\mathbf{y})&=\frac{1}{4\pi |\mathbf{x}-\mathbf{y}|}\\
&=\frac{1}{4\pi\sqrt{r^2+r'^2-2rr'\cos{(\theta-\theta')}+(z-z')^2}}\\
&=\sum_{n\in\mathbb{Z}}\frac{e^{in\theta}}{\sqrt{2\pi}}k_n(r,z,r',z')\\
\mbox{where }k_n(r,z,r',z')&=\frac{1}{\sqrt{8\pi^3rr'}}Q_{n-\frac{1}{2}}\bigg(\frac{r^2+(r')^2+(z-z')^2}{2rr'}\bigg)
\end{align}
and $Q_{n-\frac{1}{2}}$ is the half-integer order Legendre function of the second kind.

Notice that $k_n$ does not depend only on the difference between each coordinate, e.g. $(r-r')$. This has implications for the translation operators we construct for the KIFMM which is discussed in $\S3.2.4$.

\subsection{Full algorithm}
Recall that in the 3D KIFMM, we needed to solve surface integral equations like:
\begin{align}
\mbox{S2M: }&\int_{\mathbf{y}^{B,u}}{k(\mathbf{x},\mathbf{y})}\phi^{B,u}{(\mathbf{y})}d\mathbf{y}=\sum\limits_{i\in I_s^B} k(\mathbf{x},\mathbf{y}_i)\phi_i\mbox{ for all }\mathbf{x}\in\mathbf{x}^{B,u}
\end{align}
We can write $(15)$ as a sequence of 2D equations, $(16)$, by using the Fourier representation explained in $\S2.2$. If this is unclear, equation $(15)$ is to $(16)$ as equation $(1)$ is to $(3)$. In calculations below, I will skip this derivation and just state the sequences of 2D equations.

Now that we have our kernel $k_n$, we proceed through the standard 2D KIFMM algorithm for the sources on the 2D generating curve $\gamma$ \textbf{for each} $n\in[-N,-N+1,\dots,N]$ where $N$ is the truncation parameter chosen earlier in $(10)$. The KIFMM algorithm is very similar to the FMM algorithm described in \cite{CGR}, apart from how the equivalent densities are represented, and how the translation operators are computed. As mentioned earlier, rather than by multipole expansions, the potential due to sources in a box is matched to an equivalent density at discretization points on a surface enclosing the box. In 2D, these surfaces are circles with radii prescribed in \cite{YBZ}. To compute these equivalent densities, we will need to discretize several integral operators on different surfaces, which is explained in $\S3.2.2$.

In the following equations, $\mathbf{x}=(r,z)$ and $\mathbf{y}=(r',z')$. Also, please note the seemingly out-of-place $r'$ term under each integral, and recall that this is actually part of the integral operator $K_n$ as in $(3)$.
\subsubsection{Equivalent densities}
After partitioning the hierarchical tree with no more than a prescribed number of sources in each box, compute the upward equivalent density for each leaf box. Similar to the \textbf{S2M} step in the FMM, solving the following equation for $\phi^{B,u}$ gives the upward equivalent density for a box $B$ in the KIFMM:
\begin{align}
\mbox{S2M: }&\int_{\mathbf{y}^{B,u}}{k_n(\mathbf{x},\mathbf{y})}\phi^{B,u}_n{(\mathbf{y})}r'd\mathbf{y}=\sum\limits_{i\in I_s^B} k_n(\mathbf{x},\mathbf{y}_i)\phi_ir'_i\mbox{ for all }\mathbf{x}\in\mathbf{x}^{B,u}\\
\mbox{Discretized S2M: }&M_n\phi^{B,u}_n=q_n^{B,u}
\end{align}
where in $(17)$, $M_n$ is the matrix of kernel evaluations of pairs of discretization points on the upward check surface and upward equivalent surface of a box $B$, $q_n^{B,u}$ is the upward check potential, and the densities at the actual source points in the box are $\phi_i$. This is similar to $(7)$, and this process is illustrated in Figure 3. Once the upward equivalent density is computed for each leaf box, we use translation operators to find the upward and downward equivalent densities for every other box.

\begin{figure}[!ht]
\begin{center}
%\includegraphics[scale=0.5]{ued-curve}
%\includegraphics[scale=0.5]{ded-curve}
\end{center}
\caption{Left: The upward check (blue) and equivalent (red) surfaces of a box used to compute the upward equivalent density due to source densities (black) in the box. Right: The downward check (blue) and equivalent (red) surfaces of a box used to compute the downward equivalent density due to sources densities (black) in the far field on the box. In the algorithm, we never actually directly compute a downward equivalent density.}
\end{figure}

\subsubsection{Translation Operators}

Translation operators limit the number of equivalent densities we need to compute directly by translating upward equivalent densities of the leaf boxes to upward and downward equivalent densities of all other boxes. In the KIFMM in particular, the translation operators in their discretized form are matrices that translate an equivalent density from one box to that of another. They are a pre-computation, as they are constant regardless of the source distribution.

In the previous step, we computed the upward equivalent density for each leaf box. The \textbf{M2M} operator translates the upward equivalent density from a leaf box $A$ to the upward equivalent density of its parent box $B$. Solving the following equation for $\phi_n^{B,u}$ gives the M2M operator.
\begin{align}
\mbox{M2M: }&\int_{\mathbf{y}^{B,u}}{k_n(\mathbf{x},\mathbf{y})}\phi^{B,u}_n{(\mathbf{y})}r'd\mathbf{y} = \int_{\mathbf{y}^{A,u}}{k_n(\mathbf{x},\mathbf{y})}\phi^{A,u}_n{(\mathbf{y})}r'd\mathbf{y}\mbox{ for all }\mathbf{x}\in\mathbf{x}^{B,u}\\
\mbox{Discretized M2M: }&M_n^B\phi^{B,u}_n=M_n^A\phi^{A,u}_n
\end{align}
where $M_n^A$ is the matrix of kernel evaluations of pairs of discretization points on the upward check surface of box $B$ and the upward equivalent surface of box $A$, and $M_n^B$ is the matrix of kernel evaluations of pairs of discretization points on the upward check surface of box $B$ and the upward equivalent surface of box $B$. Figure 4 gives a graphical representation of this step.

So the M2M translation operator is:

\begin{align}
T^{M2M} &= (M_n^B)^{-1}M_n^A\\
&= \begin{pmatrix}
  K(\mathbf{x_1},\mathbf{y_1}) & \cdots & K(\mathbf{x_1},\mathbf{y_m})  \\
  \vdots  & \ddots & \vdots  \\
  K(\mathbf{x_m},\mathbf{y_1}) & \cdots & K(\mathbf{x_m},\mathbf{y_m}) 
 \end{pmatrix}^{-1}\begin{pmatrix}
  K(\mathbf{x_1},\mathbf{y_1}) & \cdots & K(\mathbf{x_1},\mathbf{y_m})  \\
  \vdots  & \ddots & \vdots  \\
  K(\mathbf{x_m},\mathbf{y_1}) & \cdots & K(\mathbf{x_m},\mathbf{y_m}) 
 \end{pmatrix}
\end{align}


\begin{figure}[!ht]
\begin{center}
%\includegraphics[scale=0.5]{M2M-curve}
\end{center}
\caption{The upward equivalent density due to the source densities (black) in the child box computed from the upward equivalent (red) and check (blue) surfaces of the child box is translated to the upward equivalent density of the parent box, checking against its upward check (blue) and equivalent (red) surfaces.}
\end{figure}

After repeating that step, every box now has an upward equivalent density. The \textbf{M2L} operator translates the upward equivalent density from a non-leaf box $A$ to the downward equivalent density of a box $B$ on the same level.
\begin{align}
\mbox{M2L: }&\int_{\mathbf{y}^{B,d}}{k_n(\mathbf{x},\mathbf{y})}\phi^{B,d}_n{(\mathbf{y})}r'd\mathbf{y} = \int_{\mathbf{y}^{A,u}}{k_n(\mathbf{x},\mathbf{y})}\phi^{A,u}_n{(\mathbf{y})}r'd\mathbf{y}\mbox{ for all }\mathbf{x}\in\mathbf{x}^{B,d}\\
\mbox{Discretized M2L: }&M_n^B\phi^{B,d}_n=M_n^A\phi^{A,u}_n
\end{align}
where $M_n^A$ is the matrix of kernel evaluations of pairs of discretization points on the downward check surface of box $B$ and the upward equivalent surface of box $A$, and $M_n^B$ is the matrix of kernel evaluations of pairs of discretization points on the downward check surface of box $B$ and the downward equivalent surface of box $B$.

\begin{figure}[!ht]
\begin{center}
%\includegraphics[scale=0.5]{M2L-curve}
\end{center}
\caption{The upward equivalent density due to the source densities (black) in a box computed from the upward equivalent (red) and check (blue) surfaces of the child box is translated to the downward equivalent density of another box on the same level, checking against its upward check (blue) and equivalent (red) surfaces.}
\end{figure}

After repeating that step, every non-leaf box has a downward equivalent density. The \textbf{L2L} operator translates the downward equivalent density of a non-leaf box $A$ to the downward equivalent density of a child box $B$.
\begin{align}
\mbox{L2L: }&\int_{\mathbf{y}^{B,d}}{k_n(\mathbf{x},\mathbf{y})}\phi^{B,d}_n{(\mathbf{y})}r'd\mathbf{y}=\int_{\mathbf{y}^{A,d}}{k_n(\mathbf{x},\mathbf{y})}\phi^{A,d}_n{(\mathbf{y})}r'd\mathbf{y}\mbox{ for all }\mathbf{x}\in\mathbf{x}^{B,d}\\
\mbox{Discretized L2L: }&M_n^B\phi^{B,d}_n=M_n^A\phi^{A,d}_n
\end{align}
where $M_n^A$ is the matrix of kernel evaluations of pairs of discretization points on the downward check surface of box $B$ and the downward equivalent surface of box $A$, and $M_n^B$ is the matrix of kernel evaluations of pairs of discretization points on the downward check surface of box $B$ and the downward equivalent surface of box $B$. After repeating this step, every box now has an upward and downward equivalent density.

\begin{figure}[!ht]
\begin{center}
%\includegraphics[scale=0.5]{L2L-curve}
\end{center}
\caption{The downward equivalent density due to the source densities (black) in a parent box computed from the upward equivalent (red) and check (blue) surfaces of the parent box is translated to the downward equivalent density of a child box, checking against its upward check (blue) and equivalent (red) surfaces.}
\end{figure}

Once we have these upward and downward equivalent densities for every $n\in[-N,\dots,N]$, then we can reconstruct the 3D equivalent densities $\phi_{approx}$ by summing as in $(10)$. This summing can be accelerated via the FFT. Having these equivalent densities in 3D, we can complete the last step of the KIFMM algorithm by evaluating the far- and then near-field interactions.

Note that all requirements for smoothness and uniqueness of the solution to the integral equations in the KIFMM listed in $\S3.1.5$ of \cite{YBZ} (mostly pertaining to the sizes and positions of the surfaces) are satisfied here.

\subsubsection{Discretization details}
The equations in $\S3.2.1$ are discretized using $p$ points on the equivalent and check surfaces. $p$ is constant for all boxes, and the choice of $p$ determines the error level of the computations.This discretization requires two steps. First, the right hand side is the evaluation of a check potential. This step checks that the potential represented by the equivalent density and the actual source densities are the same to all boxes in the far field. Then on the left hand side, we invert a Dirichlet-type boundary integral equation to obtain the equivalent density. To stably solve this equation, as in \cite{YBZ}, we use Tikhonov regularization with regularization parameter $\alpha=10^{-12}$. Essentially, each translation is simply applying a series of matrices.

For example, the M2L operator is the matrix $T^{M2L}_n$ for a mode $n$ is obtained by solving $(20)$:
\begin{align}
\phi^{B,d}_n&=\bigg[\big[\alpha I +(M_n^B)^TM_n^B\big](M_n^B)^TM_n^A\bigg]\phi^{A,u}_n\\
\implies T^{M2L}_n &= \bigg[\big[\alpha I +(M_n^B)^TM_n^B\big](M_n^B)^TM_n^A\bigg]
\end{align}

\subsubsection{Non-uniform translation operators}
It's important to note that in the original KIFMM, translation operators only differ based on relative position and level in the hierarchical tree. This is because the kernels used there only depended the relative difference between coordinate values, e.g. $(r-r')$. As mentioned in $\S3.1$, for the single-layer Laplace kernel, however, we are not so lucky and so we need to look for some other relationship between the translation operators.

In particular, we notice from $(3)$ and $(14)$ that each integrand
\begin{align}
k_n(r,z,r',z')r'=\sqrt{\frac{r'}{8\pi^3r}}Q_{n-\frac{1}{2}}\bigg(\frac{r^2+(r')^2+(z-z')^2}{2rr'}\bigg)
\end{align}
depends on $\frac{r^2+(r')^2+(z-z')^2}{2rr'}$ and a constant $\sqrt{\frac{r'}{8\pi^3r}}$. Since the translation operators are just matrices of kernel evaluations, they depend on these same values.

For example, take $\frac{r^2+(r')^2+(z-z')^2}{2rr'}$ and $\sqrt{\frac{r'}{8\pi^3r}}$ as the terms an M2L operator from one box to a box horizontally one unit away depends on. If we wanted to know the M2L operator from that same box to a box horizontally two units away, we could substitute $r'+1$ for $r'$ and see that now the operator depends on $\frac{r^2+(r')^2+2r'+1+(z-z')^2}{2rr'+2r}$ and $\sqrt{\frac{r'}{8\pi^3r}+\frac{1}{8\pi^3r}}$. 

One relationship we have noticed is that M2M operators, holding their global and relative positions constant, were the same regardless of the size of the box. That is, if you had child, parent, and grandparent boxes such that the bottom left corner was the same for every box, the child was the bottom left quadrant of the parent, and the parent was the bottom left quadrant of the grandparent, then the M2M operator from the child to the parent was the same as from the parent to the grandparent. We suspect this may be the same for L2L as well.

In future work, we hope to find concrete relationships between the translation operators to minimize the pre-computation. Otherwise we'd have to compute every translation operator in the entire computational domain which would be expensive.

\section{Implementation and Future Work}
Currently, we are able to produce all equivalent densities and translation operators with programs written and tested in Python, using the single-layer 3D Laplace kernel's Fourier modes. We have tested with several non-circular axisymmetric surfaces of revolution.

Much of the future work to do is in fully implementing the algorithm in Python. One key piece of this will be accelerating the matrix operations in $\S3$ using singular value decomposition (SVD) and other acceleration techniques described in \cite{MV}. SVD is applicable in this scenario since all interactions in the far field are low rank for many kernels.

In addition, in $\S3.1$ we mentioned that this strategy is currently only applicable to the single-layer Laplace kernel because of the convenient analytic determination of its Fourier modes. We hope to be able to apply this strategy to other kernels as well by discretizing them as in \cite{YYM}.

Other implementation work that needs to be done is testing various kernels, surfaces, and potentials, to perhaps find a methodical way to determine the optimal number of discretization points $p$ and the truncation parameter $N$.

Lastly, once the algorithm is completely accelerated and applicable to different kernels, performing error and complexity analysis will be an important part of its evaluation as a fast method.

% CHAPTER 2!
\chapter{Kernel-independent FMM\label{chap:two}}

In this chapter, \ldots

\section{Background\label{sec:hypothesis}}


% CHAPTER 3!
\chapter{Black-box FMM\label{chap:three}}

In this chapter, we discuss the black-box FMM.

\section{Motivation\label{sec:motivation}}

The bbFMM is ideal for complicated kernels, perhaps without translation invariance, or maybe having no explicit analytic form or only defined at a certain number of points. Maybe it has a very complicated analytic form that isn't easy to work with in the standard FMM or KIFMM. The bbFMM only requires the evaluation of the kernel at specific points.

These points are Chebyshev nodes, which are the roots of $T(x)=$. The formula for nodes on $[a,b]$ is: $x=$. This is just in one dimension, points on an interval. For two dimensions, consider two axes $r$ and $z$, and the Chebyshev nodes on each interval. The nodes in the box these intervals create are all of the possible $(r,z)$ pairs of the nodes in each respective interval.


% CHAPTER 4!
\chapter{Application to Modal Green's Functions\label{chap:four}}

In this chapter, \ldots

\section{M2M\label{sec:m2m}}

\section{M2L\label{sec:m2l}}

\section{L2L\label{sec:l2l}}

% CHAPTER 5!
\chapter{Numerical Results\label{chap:five}}

In this chapter, we provide numerical results and graphs for several cases to demonstrate the validity of our computational complexity. Also to demonstrate the raw speed of the computation.

\section{Summary\label{sec:numerics}}

In particular, we fix the root level of the computational domain as the box $[-1,1]\times [-1,1]$ in two-dimensional $(r,z)$-space, and vary the number of levels in the computational domain, and the number of Chebyshev nodes in each box. Note that the number of Chebyshev nodes is the same for every box, regardless of size or level. Note also that we have just provided reasoning for why we're able to simply consider one root level of the computational domain. We'll have the same number of operators to compute, and so the computational complexity doesn't depend on the size or location of this root box.

% CHAPTER 6!
\chapter{Future Work\label{chap:six}}

In this chapter, we discuss work that needs to be done. Ultimately, the translation operators are the core of any kernel-independent fast multipole method. They will work with any density distribution, so once that is specified, we can add a hierarchical tree and complete the algorithm.

One potential problem is with the evaluation of $k_n$ at certain Chebyshev nodes with either $r$-value equal to $0$. This will be undefined since $\chi = \frac{r^2+(r')^2+(z-z')^2}{2rr'}$. For example, this would occur if we have the box $[-a,a]\times[-a,a]$, such that a node is the origin. However, perhaps only the right half-place need be considered because if we're considering axisymmetric three-dimensional distributions then in two dimensions, we are considering a generative curve only in the right half-plane.

\section{Background\label{sec:hypothesis}}

% BIBLIOGRAPHY!
\begin{thebibliography}{99}\addcontentsline{toc}{chapter}{Bibliography}

\bibitem{CGR} Cheng, H., Greengard, L., Rokhlin, V., \emph{A Fast Adaptive Multipole Algorithm in Three Dimensions}. Journal of Computational Physics, 155, (1999), 468-498.

\bibitem{FD} Fong, W., Darve, E., \emph{A black-box fast multipole method}. Journal of Computational Physics.

\bibitem{HMY} Hao, S., Martinsson, P.G., Young, P., \emph{An efficient and highly accurate solver for multi-body acoustic scattering problems involving rotationally symmetric scatterers}. Computers and Mathematics with Applications, 69, (2015), 304-318.

\bibitem{MV} Martinsson, P.G., Rokhlin, V., \emph{An accelerated kernel-independent fast multipole method in one dimension}. SIAM Journal of Scientific Computing, Vol. 29, No. 3, (2007), 1160-1178.

\bibitem{YBZ} Ying, L., Biros, G., Zorin, D., \emph{A kernel-independent adaptive fast multipole method algorithm in two and three dimensions}. Journal of Computational Physics, 196, (2004), 591-626.

\bibitem{YYM} Young, P., Yao, S., Martinsson, P.G., \emph{A high-order Nystr{\"o}m discretization scheme for boundary integral equations defined on rotationally symmetric surfaces}. Journal of Computational Physics, 231, (2012), 4142-4159.

\end{thebibliography}

%\input{chap2} % further chapters -- change file names to meaningful things...
%\input{chap3}
%\part{Second Part\label{part:two}}%
%\input{chap4}
%\input{chap5}
%\input{chap6}
%%%%% Appendices start %%%%%%%%%%%%%%%%
%% Comment out the following line if your thesis has no appendix
%\appendix
%\chapter{One more comment\label{chap:append}}

This is an appendix.

%\chapter{One more comment\label{chap:append}}

%This is an appendix.
%% Note: If your thesis has more than one appendix, NYU requires a "list of
%% appendices" page before the body of the thesis. I don't provide the tools
%% to create that here, so you're on your own for that one... Sorry.
%\input{app2}
%%%% Input bibliography file %%%%%%%%%%%%%%%
%%% Bibliography. I didn't use BibTeX, so you're on your own if you'd like to do so.
%
\begin{thebibliography}{99}\addcontentsline{toc}{chapter}{Bibliography}
\bibitem{JBC91}J.~B.~Conway, \emph{Functions of One Complex
    Variable~I}. Second edition. Springer-Verlag, Graduate
    Texts in Mathematics~\textbf{11}, 1991.
\end{thebibliography}

%\begin{thebibliography}{99}\addcontentsline{toc}{chapter}{Bibliography}
%\bibitem{JBC91}J.~B.~Conway, \emph{Functions of One Complex
%    Variable~I}. Second edition. Springer-Verlag, Graduate
%    Texts in Mathematics~\textbf{11}, 1991.
%\end{thebibliography}
\end{document}
