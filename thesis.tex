%% NYU PhD thesis format. Created by Jos� Koiller 2007--2008.

%% Use the first of the following lines during production to
%% easily spot "overfull boxes" in the output. Use the second
%% line for the final version.
%\documentclass[12pt,draft,letterpaper]{report}
\documentclass[12pt,letterpaper]{report}

%% Replace the title, name, advisor name, graduation date and dedication below with
%% your own. Graduation months must be January, May or September.
\newcommand{\thesistitle}{Translation Operators for Translation-Variant Modal Green's Functions}
\newcommand{\thesisauthor}{Victor Churchill}
\newcommand{\thesisadvisor}{Professor Michael O'Neil}
\newcommand{\graddate}{May 2016}
%% If you do not want a dedication, scroll down and comment out
%% the appropriate lines in this file.
\newcommand{\thesisdedication}{To Mary.}

%% The following makes chapters and sections, but not subsections,
%% appear in the TOC (table of contents). Increase to 2 or 3 to
%% make subsections or subsubsections appear, respectively. It seems
%% to be usual to use the "1" setting, however.
\setcounter{tocdepth}{1}

%% Sectional units up to subsubsections are numbered. To number
%% subsections, but not subsubsections, decrease this counter to 2.
\setcounter{secnumdepth}{3}

%% Page layout (customized to letter paper and NYU requirements):
\setlength{\oddsidemargin}{.6in}
\setlength{\textwidth}{5.8in}
\setlength{\topmargin}{.1in}
\setlength{\headheight}{0in}
\setlength{\headsep}{0in}
\setlength{\textheight}{8.3in}
\setlength{\footskip}{.5in}

%% Use the following commands, if desired, during production.
%% Comment them out for final version.
%\usepackage{layout} % defines the \layout command, see below
%\setlength{\hoffset}{-.75in} % creates a large right margin for notes and \showlabels

%% Controls spacing between lines (\doublespacing, \onehalfspacing, etc.):
\usepackage{setspace}

%% Use the line below for official NYU version, which requires
%% double line spacing. For all other uses, this is unnecessary,
%% so the line can be commented out.
\doublespacing % requires package setspace, invoked above

%% Each of the following lines defines the \com command, which produces
%% a comment (notes for yourself, for instance) in the output file.
%% Example:    \com{this will appear as a comment in the output}
%% Choose (uncomment) only one of the three forms:
%\newcommand{\com}[1]{[/// {#1} ///]}       % between [/// and ///].
\newcommand{\com}[1]{\marginpar{\tiny #1}} % as (tiny) margin notes
%\newcommand{\com}[1]{}                     % suppress all comments.

%% This inputs your auxiliary file with \usepackage's and \newcommand's:
%% It is assumed that that file is called "definitions.tex".
%%
%% Place here your \usepackage's. Some recommended packages are already included.
%%

% Graphics:
\usepackage[final]{graphicx}
%\usepackage{graphicx} % use this line instead of the above to suppress graphics in draft copies
%\usepackage{graphpap} % \defines the \graphpaper command

% Indent first line of each section:
\usepackage{indentfirst}

% Good AMS stuff:
\usepackage{amsthm} % facilities for theorem-like environments
\usepackage[tbtags]{amsmath} % a lot of good stuff!

% Fonts and symbols:
\usepackage{amsfonts}
\usepackage{amssymb}

% Formatting tools:
%\usepackage{relsize} % relative font size selection, provides commands \textsmalle, \textlarger
%\usepackage{xspace} % gentle spacing in macros, such as \newcommand{\acims}{\textsc{acim}s\xspace}

% Page formatting utility:
%\usepackage{geometry}

%%
%% Place here your \newcommand's and \renewcommand's. Some examples already included.
%%
\renewcommand{\le}{\leqslant}
\renewcommand{\ge}{\geqslant}
\renewcommand{\emptyset}{\ensuremath{\varnothing}}
\newcommand{\ds}{\displaystyle}
\newcommand{\R}{\ensuremath{\mathbb{R}}}
\newcommand{\Q}{\ensuremath{\mathbb{Q}}}
\newcommand{\Z}{\ensuremath{\mathbb{Z}}}
\newcommand{\N}{\ensuremath{\mathbb{N}}}
\newcommand{\T}{\ensuremath{\mathbb{T}}}
\newcommand{\eps}{\varepsilon}
\newcommand{\closure}[1]{\ensuremath{\overline{#1}}}
%\newcommand{\acim}{\textsc{acim}\xspace}
%\newcommand{\acims}{\textsc{acim}s\xspace}

%%
%% Place here your \newtheorem's:
%%

%% Some examples commented out below. Create your own or use these...
%%%%%%%%%\swapnumbers % this makes the numbers appear before the statement name.
%\theoremstyle{plain}
%\newtheorem{thm}{Theorem}[chapter]
%\newtheorem{prop}[thm]{Proposition}
%\newtheorem{lemma}[thm]{Lemma}
%\newtheorem{cor}[thm]{Corollary}

%\theoremstyle{definition}
%\newtheorem{define}{Definition}[chapter]

%\theoremstyle{remark}
%\newtheorem*{rmk*}{Remark}
%\newtheorem*{rmks*}{Remarks}

%% This defines the "proo" environment, which is the same as proof, but
%% with "Proof:" instead of "Proof.". I prefer the former.
%\newenvironment{proo}{\begin{proof}[Proof:]}{\end{proof}}


%% Cross-referencing utilities. Use one or the other--whichever you prefer--
%% but comment out both lines for final version.
%\usepackage{showlabels}
%\usepackage{showkeys}


\begin{document}
%% Produces a test "layout" page, for "debugging" purposes only.
%% Comment out for final version.
%\layout % requires package layout (see above, on this same file)

%%%%%% Title page %%%%%%%%%%%
%% Sets page numbering to "roman style" i, ii, iii, iv, etc:
\pagenumbering{roman}
%
%% No numbering in the title page:
\thispagestyle{empty}
%
\begin{center}
  {\large\textbf{\thesistitle}}
  \vspace{.7in}

  by
  \vspace{.7in}

  \thesisauthor
  \vfill

\begin{doublespace}
  A dissertation submitted in partial fulfillment\\
  of the requirements for the degree of\\
  Master of Science\\
  Department of Mathematics\\
  New York University\\
  \graddate
\end{doublespace}
\end{center}
\vfill

\noindent\makebox[\textwidth]{\hfill\makebox[2.5in]{\hrulefill}}\\
\makebox[\textwidth]{\hfill\makebox[2.5in]{\hfill\thesisadvisor\hfill}}
\newpage
%%%%%%%%%%%%% Blank page %%%%%%%%%%%%%%%%%%
\thispagestyle{empty}
\vspace*{0in}
\newpage

%%%%%%%%%%%%%% Dedication %%%%%%%%%%%%%%%%%
%% Comment out the following lines if you do not want to dedicate
%% this to anyone...
\vspace*{\fill}
\begin{center}
  \thesisdedication\addcontentsline{toc}{section}{Dedication}
\end{center}
\vfill
\newpage
%%%%%%%%%%%%%% Acknowledgements %%%%%%%%%%%%
%% Comment out the following lines if you do not want to acknowledge
%% anyone's help...
\section*{Acknowledgements}\addcontentsline{toc}{section}{Acknowledgements}
%% Write your acknowledgements in this file. If you do not want to acknowledge anyone,
%% you can delete this file and comment out the corresponding part in the "thesis.tex"
%% file.
%
I would like to acknowledge the effort put into this thesis by
myself. This work could not have been done without me.

\newpage
%%%% Abstract %%%%%%%%%%%%%%%%%%
\section*{Abstract}\addcontentsline{toc}{section}{Abstract}
%% Write your abstract here.
%
Translation operators for various flavors of the Fast Multipole Method (FMM) are a minimal pre-computation for translation-invariant Green's functions and modal Green's functions. This is due to the fact that for these functions many translation operators on various levels the computational domain are the same, and so need not be repeatedly computed. However, for a translation-variant Green's function or modal Green's function this is not immediately clear. Until now one may have needed to compute all possible translation operators between all boxes in the computational domain. This thesis shows that lightening the pre-computational load of the translation operators for the FMM is possible for translation-variant modal Green's functions. Specifically we examine the application to the translation operators of the three-dimensional kernel-independent FMM for Laplace's equation.

\newpage
%%%% Table of Contents %%%%%%%%%%%%
\tableofcontents

%%%%% List of Figures %%%%%%%%%%%%%
%% Comment out the following two lines if your thesis does not
%% contain any figures. The list of figures contains only
%% those figures included withing the "figure" environment.
\listoffigures\addcontentsline{toc}{section}{List of Figures}
\newpage

%%%%% List of Tables %%%%%%%%%%%%%
%% Comment out the following two lines if your thesis does not
%% contain any tables. The list of tables contains only
%% those tables included withing the "table" environment.
\listoftables\addcontentsline{toc}{section}{List of Tables}
\newpage

%%%%% Body of thesis starts %%%%%%%%%%%%
\pagenumbering{arabic} % switches page numbering to arabic: 1, 2, 3, etc.
%% Introduction. If your thesis has no introduction, or chapter 1 is
%% meant to be the introduction, then comment out the lines below.
\section*{Introduction}\addcontentsline{toc}{section}{Introduction}
%% Write your introduction here.
%
This thesis is about the Riemann Hypothesis. We provide an
affirmative answer to the question ``If $z$ is a zero of the
zeta function and $0\le \Re(z)\le1$, then is $\Re(z)$
necessarily equal to $1/2$?''

In chapter~\ref{chap:one} we do this and that.

In the latter chapters we\ldots

%% If your thesis has different "Parts", use commands such as the following:
%\part{First Part\label{part:one}}%
\chapter{Statement of problem\label{chap:one}}

In this chapter, \ldots

\section{The Riemann Hypothesis\label{sec:hypothesis}}

Blah, blah, blah. There is nothing interesting in
figure~\ref{fig:afigure}.
\begin{figure}[htb]
  \begin{center}
    \emph{This statement is false.}
  \end{center}%
\caption[This alternate caption appears in the list of figures.]{This is the
caption that appears under the figure. It may be quite long---you wouldn't want
such a long caption to appear in the ``list of figures''.}
\label{fig:afigure}
\end{figure}

More blah, blah. There is nothing interesting about
table~\ref{tab:atable} either.
\begin{table}[htb]
\caption[Strange rules.]{For some reason unfamiliar to me,
typesetting rules require one to place captions above tables,
but below figures. Go figure.}\label{tab:atable}
  \begin{center}
    \framebox{You could put a table here. I won't.}
  \end{center}
\end{table}

\section{Another section\label{sec:two}}

Notice that the fist paragraph is indented. There's a package
to do that automatically. Blah, blah. Blah, blah, blah, blah.

%\input{chap2} % further chapters -- change file names to meaningful things...
%\input{chap3}
%\part{Second Part\label{part:two}}%
%\input{chap4}
%\input{chap5}
%\input{chap6}
%%%%% Appendices start %%%%%%%%%%%%%%%%
%% Comment out the following line if your thesis has no appendix
\appendix
\chapter{One more comment\label{chap:append}}

This is an appendix.

%% Note: If your thesis has more than one appendix, NYU requires a "list of
%% appendices" page before the body of the thesis. I don't provide the tools
%% to create that here, so you're on your own for that one... Sorry.
%\input{app2}
%%%% Input bibliography file %%%%%%%%%%%%%%%
%% Bibliography. I didn't use BibTeX, so you're on your own if you'd like to do so.
%
\begin{thebibliography}{99}\addcontentsline{toc}{chapter}{Bibliography}
\bibitem{JBC91}J.~B.~Conway, \emph{Functions of One Complex
    Variable~I}. Second edition. Springer-Verlag, Graduate
    Texts in Mathematics~\textbf{11}, 1991.
\end{thebibliography}


\end{document}
